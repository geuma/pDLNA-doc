%
% FIXED ISSUES
%

\chapter{Fixed Issues}
\label{fixedissues}

\section{DirectoryListings not working when mandatory data fields not defined}

\begin{colframeimportantnote}
\textsc{IMPORTANT NOTE:} This issue has been fixed with {\em pDLNA} in version 0.64.2.
\end{colframeimportantnote}

\subsection{Describtion}

Directory listings are not working, when {\em mandatory} data fields are not defined in the XML \verb|<FILTER>| tag of the browse request from a DLNA media renderer.\footnote{\url{github.com/geuma/pDLNA/issues/24}}.

\subsection{Workaround}

There is no workaround for this issue.

\section{Movie::Info is still used by PDLNA::Media}

\begin{colframeimportantnote}
\textsc{IMPORTANT NOTE:} This issue has been fixed with {\em pDLNA} in version 0.64.1.
\end{colframeimportantnote}

\subsection{Describtion}

The \verb|PDLNA::Media| Perl module is still using \verb|Movie::Info| Perl module, even if it is not required any more. So {\em pDLNA} will not start up if the \verb|Movie::Info| Perl module is not installed\footnote{\url{github.com/geuma/pDLNA/issues/23}}.

\subsection{Workaround}

Install the missing Perl module \verb|Movie::Info|.

\section{parsing problem if media file name contains special regex characters like '(', ')' or others}

\begin{colframeimportantnote}
\textsc{IMPORTANT NOTE:} This issue has been finally fixed with {\em pDLNA} in version 0.65.0.
\end{colframeimportantnote}

\subsection{Describtion}

This issue has been reported on a \verb|Raspberry Pi| running Linux with Perl 5.14.2 and locales configured to \verb|de_DE.UTF-8|.

The following error appears, when files with special characters (for instance brackets) are existing, which break some regular expressions in {\em pDLNA}\footnote{\url{github.com/geuma/pDLNA/issues/21}}:
\begin{lstlisting}
Issuing rollback() due to DESTROY without explicit disconnect() of DBD::SQLite::db handle dbname=/media/TREK_2/Audio/pdlna.db at ./pDLNA.pl line 63.
Thread 1 terminated abnormally: Unmatched ) in regex; marked by <-- HERE in m/^18-This Live I) <-- HERE 'm Living.mp3$/ at PDLNA/ContentLibrary.pm line 152.
\end{lstlisting}

\subsection{Workaround}

There is no workaround for this issue. Or would you like to rename your files?

\section{Use of uninitialized value \$request\_line in string ne at /PDLNA/HTTPServer.pm line xxx.}

\begin{colframeimportantnote}
\textsc{IMPORTANT NOTE:} This issue has been fixed with {\em pDLNA} in version 0.62.0.
\end{colframeimportantnote}

\subsection{Describtion}

Apparently {\em pDLNA} sometimes starts to print, for unknown reasons, the following warning message to \verb|STDOUT| infinitely\footnote{\url{github.com/geuma/pDLNA/issues/13}}:
\begin{lstlisting}
Use of uninitialized value $request_line in string ne at /PDLNA/HTTPServer.pm line xxx.
\end{lstlisting}

\subsection{Workaround}

Since, this message is only a warning message, it does not affect the functionality of {\em pDLNA}. But actually, it increases the system load. A restart of {\em pDLNA} will fix the problem temporarily, but the issue might take place again.

\section{AllowedClients auto detection on FreeBSD is defect}

\begin{colframeimportantnote}
\textsc{IMPORTANT NOTE:} This issue has been fixed with {\em pDLNA} in version 0.61.0.
\end{colframeimportantnote}

\subsection{Describtion}

Apparently {\em pDLNA} dies with the following error message on {\em FreeBSD} if {\em AllowedClients} auto detection is enabled\footnote{\url{github.com/geuma/pDLNA/issues/4}}:
\begin{lstlisting}
AllowedClients(ip_address_sv) at PDLNA/Config.pm line xxx.
\end{lstlisting}
This seems to be a problem with the usage of \verb|AllowedClients(ip_address_sv)| on {\em FreeBSD}.

\subsection{Workaround}

As a workaround you are able to define the {\em AllowedCLients} configuration parameter in the configuration file. For detailed information see section~\ref{confallowedclients}.